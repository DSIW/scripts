% Farben
\definecolor{orange}{rgb}{1,0.39,0.04}
\definecolor{dunkelblau}{rgb}{0,0.08,0.45} 
\definecolor{dunkelgrau}{gray}{0.35}
\definecolor{mittelgrau}{gray}{0.85}
\definecolor{hellgrau}{gray}{0.93}
\definecolor{hellgelb}{rgb}{1.0,1.0,0.9}

% Font ändern
\newcommand{\changefont}[3]{\fontfamily{#1} \fontseries{#2} \fontshape{#3} \selectfont}

% Abkürzungen mit korrektem Leerraum 
\newcommand{\ua}{\mbox{u.\,a.}\xspace}
\newcommand{\zB}{\mbox{z.\,B.}\xspace}
\newcommand{\dahe}{\mbox{d.\,h.}\xspace}
\newcommand{\Dahe}{\mbox{D.\,h.}\xspace}
\newcommand{\Vgl}{Vgl.\xspace}
\newcommand{\bzw}{bzw.\xspace}
\newcommand{\evtl}{evtl.\xspace}
\newcommand{\uvm}{uvm.\xspace}
\newcommand{\usw}{usw.\xspace}
\newcommand{\bzgl}{bzgl.\xspace}
\newcommand{\gdw}{gdw.\xspace}
\newcommand{\ex}{existiert\xspace}

\newcommand{\abbildung}[1]{Abbildung~\ref{fig:#1}}

\newcommand{\bs}{$\backslash$}

% erzeugt ein Listenelement mit fetter Überschrift 
\newcommand{\itemd}[2]{\item{\textbf{#1}}\\{#2}}

% einige Befehle zum Zitieren
\newcommand{\Zitat}[2][\empty]{\ifthenelse{\equal{#1}{\empty}}{\citep{#2}}{\citep[#1]{#2}}}

% zum Ausgeben von Autoren
\newcommand{\AutorName}[1]{\textsc{#1}}
\newcommand{\Autor}[1]{\AutorName{\citeauthor{#1}}}
\newcommand{\mailto}[1]{\href{mailto:#1}{#1}}

% verschiedene Befehle um Wörter semantisch auszuzeichnen
\newcommand{\NeuerBegriff}[1]{\textbf{#1}\index{#1}}
\newcommand{\NeuerBegriffUnterkategorie}[2]{\textbf{#2}\index{#1!#2}}
\newcommand{\NeuerBegriffoIndex}[1]{\textbf{#1}}

\newcommand{\Fachbegriff}[1]{\textit{#1}}

\newcommand{\Eingabe}[1]{\texttt{#1}}
\newcommand{\Code}[1]{\texttt{#1}}
\newcommand{\Datei}[1]{\texttt{#1}}

\newcommand{\Datentyp}[1]{\textsf{#1}}
\newcommand{\XMLElement}[1]{\textsf{#1}}
\newcommand{\Webservice}[1]{\textsf{#1}}

% Hier werden Bilder nicht mehr mit "Bild" benannt sondern "Abbildung"
\renewcommand{\figurename}{Abbildung}

% Neuen Befehl zum Einfügen von Bildern definieren
\newcommand{\bild}[3]{
  \begin{figure}[ht]
   \centering
      \includegraphics[width=#2]{#1}
      \caption{#3}
      \label{fig:#1}
  \end{figure}
}

% URL als Fußnote
\newcommand*{\fnurl}[1]{{\footnote{{\footnotesize \url{#1}}}}}
% Weitere "Fußnoten" (ACHTUNG: statisch!)
\newcommand{\fnurln}[1]{\raisebox{1ex}{\footnotesize{[#1]}}}

% TODO-Befehl
\newcommand{\todo}[1]{\textbf{\textsc{\color{red}{(TODO: #1)}}}}
%\newcommand{\todo}[1]{\todo{#1}}
%\newcommand{\todo}[2][]{\todo[#1]{#2}}

%\newcommand{\todoupdate}[1]{\todo[color=blue!40]{#1}}
%\newcommand{\todoupdate}[2][]{\todo[color=blue!40,#1]{#2}}
%
%\newcommand{\todoquestion}[1]{\todo[color=red!40]{#1}}
%\newcommand{\todoquestion}[2][]{\todo[color=red!40,#1]{#2}}
%
%\newcommand{\todocomment}[1]{\todo[color=green!40]{#1}}
%\newcommand{\todocomment}[2][]{\todo[color=green!40,#1]{#2}}
%
%\newcommand{\todoformat}[1]{\todo[color=yellow!40]{#1}} 
%\newcommand{\todoformat}[2][]{\todo[color=yellow!40,#1]{#2}}

% Überschriften mit Label
\newcommand{\mchapter}[1]{\chapter{#1}\label{cha:#1}}
\newcommand{\msection}[1]{\section{#1}\label{sec:#1}}
\newcommand{\submsection}[1]{\subsection{#1}\label{ssec:#1}}
\newcommand{\subsubmsection}[1]{\subsubsection{#1}\label{sssec:#1}}
\newcommand{\subsubsubmsection}[1]{\textbf{\textsf{#1}\label{ssssec:#1}}}

% Hoch- und Tiefstellen
\newcommand{\up}[2]{#1\textsuperscript{#2}}
\newcommand{\down}[2]{#1\textsubscript{#2}}
\newcommand{\ggT}[1]{\textnormal{\mbox{ggT($#1$)}}}

\renewcommand{\deg}[2]{\mbox{$\textnormal{deg}_{#1}\left(#2\right)$}}
\newcommand{\indeg}[2]{\mbox{$\textnormal{indeg}_{#1}\left(#2\right)$}}
\newcommand{\outdeg}[2]{\mbox{$\textnormal{outdeg}_{#1}\left(#2\right)$}}

% Vereinfachungen f. Mathematik

% Befehl „\ensuremath{...}“ prüft ob in Mathe-Modus, wenn ja > ok; sonst erstellen
\newcommand{\Gesetz}[1]{\text{\footnotesize{Gesetz: \textsc{#1}}}}
\newcommand{\folgt}{\newline \ensuremath{\stackrel{Folge}{\Longrightarrow}}\xspace}
\newcommand{\ufkt}[1]{\ensuremath{#1^{-1}}}
\newcommand{\equivklasse}[1]{\ensuremath{\lek #1 \rek_\sim}}
% \Longleftrightarrow[60]

% Klammern
\newcommand{\lrk}{\left(}
\newcommand{\rrk}{\right)}
\newcommand{\lgk}{\left\{}
\newcommand{\rgk}{\right\}}
\newcommand{\lek}{\left[}
\newcommand{\rek}{\right]}

% boolsche Werte
\newcommand{\true}{\textsc{(True)}}
\newcommand{\false}{\textsc{(False)}}

% Zahlenraeume
\newcommand{\fline}{\bigskip\noindent}
\newcommand{\field}[1]{\ensuremath{\mathbb{#1}}\xspace}
\newcommand{\N}{\field{N}}
\newcommand{\Z}{\field{Z}}
\newcommand{\Q}{\field{Q}}
\newcommand{\R}{\field{R}}
\newcommand{\C}{\field{C}}
\newcommand{\K}{\field{K}}
% Das Differential-d, aus der Klasse bgteubner
\makeatletter
\let\origd=\d
\renewcommand*\d{
  \relax\ifmmode
    \mathrm{d}%
  \else
    \expandafter\origd
  \fi
}\makeatother
\newcommand{\expo}[1]{\exp\left\{{#1}\right\}}
\newcommand{\series}[1]{\sum_{#1}^{\infty}}
\newcommand{\stetig}[2]{\mathscr{C}(#1,#2)}
\newcommand{\topologie}{\mathfrak{T}}

% Einheiten
%\newcommand{\degree}{\ensuremath{^\circ}}
\newcommand{\gradc}[1]{\ensuremath{#1\,^{\circ}\mathrm{C}}}

% Symbole
%\newcommand{\entspricht}{\ensuremath{\stackrel{\scriptscriptstyle\wedge}{=}}\xspace}
\newcommand{\entspricht}{\ensuremath{\triangleq}\xspace}

% Makro für Tasten
% \Taste{x} für eine Taste und
% \TasteZ(wei){x}{y}, \TasteD(rei){x}{y}{z} ... für Kombinationen
% Benutzung: \TasteZ{Strg}{Q} für Tastenkombination Strg+q
\newcommand*{\Taste}[1]{\keystroke{#1}}
% Makro für Tasten
%\newcommand*{\Taste}[1]{\setlength{\fboxsep}{1pt}\fbox{\fullh{}#1}}
\newcommand*{\TasteE}[1]{\Taste{\setlength{\fboxsep}{1pt}#1}}
\newcommand*{\TasteZ}[2]{\Taste{#1}\,+\,\Taste{#2}}
\newcommand*{\TasteD}[3]{\Taste{#1}\,+\,\Taste{#2}\,+\,\Taste{#3}}
\newcommand*{\TasteV}[4]{\Taste{#1}\,+\,\Taste{#2}\,+\,\Taste{#3}\,+\,\Taste{#4}}
\newcommand*{\TasteA}[8]{\Taste{#1}\,+\,\Taste{#2}\,+\,\Taste{#3}\,+\,\Taste{#4}\,+\,\Taste{#5}\,+\,\Taste{#6}\,+\,\Taste{#7}\,+\,\Taste{#8}}

% Exceptions
\newcommand{\Exception}[1]{\texttt{#1Exception}}
