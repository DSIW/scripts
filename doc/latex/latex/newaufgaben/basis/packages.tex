% Einstellungen der Seitenränder
\usepackage[left=2.5cm,right=2.5cm,top=2cm,bottom=3cm,includeheadfoot]{geometry}

% Pakete

% Schrift
\usepackage[ngerman]{babel} % neue Rechtschreibung
\usepackage[utf8]{inputenc}	% Umlaute ermöglichen
\usepackage[T1]{fontenc} % Font ändern
% Times/Arial-Schriftarten:
%\usepackage{mathptmx} 
%\usepackage[scaled=.92]{helvet}
%\usepackage{courier}
% Schrift Palatino
%\usepackage[osf]{mathpazo}
%\linespread{1.05}
% Schrift Libertine
%\usepackage[osf]{libertine}

% Grafiken
\usepackage{rotating}       % drehbare Bilder/Texte/Tabellen
\usepackage{pdfpages}       % PDF einbinden
\usepackage{graphicx}       % Grafiken einfügen
%\usepackage{subfig}					% Grafiken einfügen
%\usepackage[subfigure]{tocloft}
\usepackage{capt-of}        % Ermöglicht das Beschriften von Tabellen ohne figure-Umgebung
\usepackage{subfigure}	    % Unterbilder
%\usepackage[nooneline]{subfigure}
%\usepackage{tikz}						% Zeichnungen
\usepackage{keystroke}			% Tastatur
\usepackage{listings}       % Listings -> Quellcodedarstellung für viele verschiedene Sprachen
\usepackage{picinpar}         % verwenden von umflossenen Grafiken
% Einbindung von allen Grafikformaten möglich
\usepackage{pst-pdf}        % pstricks und ps grafiken in pdflatex
\usepackage{pst-all}
\usepackage{pstricks-add}
% für Fließumgebungen
% Platzierung H zwingt LaTeX Fließumgebungen genau an diese Stelle zu setzen
\usepackage{float}
\usepackage{wrapfig}
\usepackage{moreverb}		    % erweiterte verbatim Umgebung
\usepackage{pdfpages}		    % Einbinden von PDF Seiten aus PDF Dokument
\usepackage{cmap}		        % to make the PDF files "searchable and copyable" in pdf viewer
%\usepackage{header}         % Beschriftung der Abbildungen und Tabellen nach Kapitelzugehörigkeit,

% Tabellen
\usepackage{multirow}		    % mehrere Zeilen in einer Tabelle zusammenfassen
\usepackage{ltxtable}		    % longtable und tabularx vereint in einer Tabelle, lädt automatisch beide Pakete

% Mathematik
\usepackage{amsmath,amsfonts, amssymb, amstext, amsthm, shadethm}
%\usepackage{mathabx} % Mathematische Symbole
\usepackage{nicefrac}		    % Darstellung eines Bruchs im Fließtext; Aussehen zB 1/4
\usepackage{textcomp}		    % einige Text-Mode Mathe Symbole, wie zB Mal-Zeichen x (siehe The Comprehensive LaTeX Symbol List)

% Symbole
\usepackage{units}		      % Einheiten komfortabel darstellen \unit[Zahlenwert]{Einheit}
\usepackage{gensymb}	    	% celsius, micro, ohm, perthousand, degree in Mathe und Textmodus
\usepackage{eurosym}        % Euro-Sybol

% Layout
\usepackage{microtype}	    % Verbessert Textsatz
\usepackage{fixltx2e}		    % Verbessert einige Kernkompetenzen von LaTeX2e

\usepackage{lmodern}		    % skalierbare Schriftfamilie "Latin Modern" (default: bitmapped "Computer Modern")
\usepackage{fancyhdr}       % Kopf- und Fußzeilen, speziell formatierbar, mit Trennlinien
\usepackage{booktabs}       % laden der booktabs.sty für Querlinien in Tabellen (\toprule, \midrule, \bottomrule)
\usepackage{multicol}				% Spalten
\usepackage{ulem}           % Packet für Textunterstreichungen
\usepackage{caption}        % Beschriftungspaket (hier: für Beschriftung der Anlagen mit \caption[]{...} ohne Eintrag ins Abb.-verz. zu erzeugen)
\usepackage{paralist}				% weitere Listen-Umgebungen: compact|aspara|inpara: compactitem, compactenum, inparaenum
\usepackage{caption}		    % viele caption Formatierungsmöglichkeiten
\usepackage{chngcntr}       % Zum fortlaufenden Durchnummerieren der Fußnoten
\usepackage{setspace}       % Zeilenabstand definieren
%\usepackage[activate]{pdfcprot} % Optischer Randausgleich (Symbole werden leicht weiter rechts ausgerichtet.)
\usepackage{marginnote}     % Randbemerkung
\usepackage{ellipsis}		    % Korrigiert den Weißraum um Auslassungspunkte
\usepackage[square]{natbib} % Wichtig für korrekte Zitierweise

% Farben
\usepackage{color, xcolor}  % Farbe benutzen
\usepackage{colortbl}       % Verwenden vordefinierter Farben

%% Sonstiges
\usepackage{xspace}   	    % Leerzeichen auch bei parameterlosen Befehlen
\usepackage{blindtext}      % Blindtext zum Testen von Textausgaben
%\usepackage[german, colorinlistoftodos]{todonotes}			% TODO verwalten - BENÖTIGT ggf.: ifthen, tikz, xkeyval, calc, xcolor

% undefinierbar
%\usepackage[hang,bottom]{footmisc}
%\usepackage[active,tightpage]{preview}

%\usepackage[norefpage,intoc,german]{nomencl}

%Verweise / Verzeichnisse
\usepackage[% PDF-Optionen
% 	%pdftex,
% 	%bookmarks,
% 	%bookmarksnumbered=true,
% 	%bookmarksopen=true,
% 	%breaklinks=true,
	pdftitle={\titel},
	pdfauthor={\autor},
	pdfcreator={\autor},
	pdfsubject={\titel},
	pdfkeywords={\titel\ \untertitel},
	colorlinks=true,
% 	%linkcolor=red, % einfache interne Verknüpfungen
% 	%linktoc=all,
	linkcolor=black,%
% 	anchorcolor=black,% Ankertext
% 	citecolor=green, % Verweise auf Literaturverzeichniseinträge im Text
% 	filecolor=magenta, % Verknüpfungen, die lokale Dateien öffnen
% 	menucolor=black, % Acrobat-Menüpunkte
% 	%pagecolor=red, % Verknüpfungen zu anderen Seiten
% 	%urlcolor=cyan,
	urlcolor=black,%
% %	backref, % inkompatibel mit cleveref
% 	%pagebackref=true,
% 	%plainpages=true,% zur korrekten Erstellung der Bookmarks
% 	%pdfpagelabels=true,% zur korrekten Erstellung der Bookmarks
% 	%hypertexnames=false,% zur korrekten Erstellung der Bookmarks
% 	%linktocpage %Seitenzahlen anstatt Text im Inhaltsverzeichnis verlinken
]{hyperref}
\usepackage{lastpage}      				% Seiteanzahl f. Fusszeile
\usepackage[ngerman]{varioref}		% Relative Verweisangaben (auf vorheriger/nächster Seite)
\usepackage[ngerman]{cleveref}		% Verweis - WICHTIG: NACH HYPERREF EINBINDEN!
\usepackage{acronym}		    % Abkürzungsverzeichnis --> nur verwendete Abkürzungen listen [printonlyused]
\usepackage{bibgerm}		    % deutsches Literaturverzeichnis
\usepackage{url}            % URL
\usepackage{hypbmsec}

%\usepackage{index}          % Index
\usepackage{makeidx}        % Für Index-Ausgabe (\printindex)
%\usepackage{minitoc}        % zwischeninhaltsverzeichnis (hier: anlagenverzeichnis)
%\usepackage{mtcoff}         % aktivieren für minitoc "AUS"

% Symbolverzeichnis
% 	makeindex.exe %Name%.nlo -s nomencl.ist -o %Name%.nls
%		Die Definitionen sind ausgegliedert in die Datei Abkuerzungen.tex.
\usepackage[intoc]{nomencl}
  \let\abbrev\nomenclature
  \renewcommand{\nomname}{Glossar und Abkürzungsverzeichnis}
  \setlength{\nomlabelwidth}{.25\hsize}
  \renewcommand{\nomlabel}[1]{#1 \dotfill}
  \setlength{\nomitemsep}{-\parsep}
