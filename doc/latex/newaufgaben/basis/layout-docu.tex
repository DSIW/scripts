% Schriftgröße, Layout, Papierformat, Art des Dokumentes
%\documentclass[12pt,oneside,a4paper]{scrartcl}

% Zeilenabstand definieren
%\usepackage{mathpazo}
%\linespread{1.5}
%\renewcommand{\baselinestretch}{1.3}
%\onehalfspacing

% Kein Einzug bei Absatzanfang
\setlength{\parindent}{0cm}

% Kopf- und Fußzeile
\pagestyle{fancy}
\fancyhf{}

% Aussehen von Fußnoten (wird nur für URLs genutzt)
\renewcommand{\thefootnote}{{\footnotesize [\arabic{footnote}]}}

% Bilder befinden sich im folgenen Pfad.
\graphicspath{{grafiken/}} % Dort liegen die Bilder des Dokuments

% Kopfzeile links bzw. innen
\fancyhead[L]{\modul}
% Kopfzeile rechts bzw. außen
\fancyhead[R]{\titel}
% Linie oben
\renewcommand{\headrulewidth}{0.5pt}

% Fußzeile links bzw. innen
\fancyfoot[L]{\autor\ (\matrikelnr)\\\autormailstudium}
% Fußzeile mittig
\fancyfoot[C]{Seite \thepage{} von \pageref{LastPage}}
% Fußzeile rechts bzw. außen
\fancyfoot[R]{\datum \\ {\scriptsize Version: \version}}
% Linie unten
\renewcommand{\footrulewidth}{0.5pt}

% Quellenangaben in eckige Klammern fassen
\bibpunct{[}{]}{;}{a}{}{,~}

% Quellcode (Listing)
\lstloadlanguages{C,JAVA}
\lstset{
	numbers=left,
	numberstyle=\tiny,
	stepnumber=1,
	numbersep=7pt
}
\lstset{language=Java,
	lineskip=1pt,				% Zeilenabstand
	breakatwhitespace=true,
	title=\lstname,
	language={},                % keine Sprache definiert
    basicstyle=\small\ttfamily, % Schriftstil
    %columns=fixed,              % fest Spaltenbreite
    tabsize=2,                  % Tabs haben Breite von 2
    frame=single,               % einfacher Rahmen
    framesep=1pt,               % Abstand des Rahmens
    framerule=0.5pt,            % Linienstaerke des Rahmens
    backgroundcolor=\color{hellgrau},	% Hintergrundfarbe
    rulecolor=\color{dunkelgrau},  % Farbe der Rahmenlinie
    extendedchars=true,         % erweiterter Zeichensatz (geht nicht)
    showspaces=false,           % Leerzeichen nicht extra darstellen
    showstringspaces=false,     % Leerzeichen in String nicht extra darstellen
    breaklines=true,            % automatischen Umbruch aktivieren
    %breakindent=0pt,            % Einrueckung nach Umbruch
    prebreak=\mbox{$\ \curvearrowright$},     % Pfeil als Umbruchzeichen
    xleftmargin=1.8pt,          % linker Abstand vom Rand (= framesep)
    xrightmargin=1.8pt,         % rechter Abstand vom Rand  (= framesep)
    aboveskip=1em,              % Abstand vor einer Box
    belowskip=0em,              % Abstand nach einer Box
    %captionpos=b,               % Position des Untertitels
  	%keywordstyle=\bfseries\color{darkred},
  	%stringstyle=\color{darkblue},
  	%commentstyle=\itshape\color{darkgreen},
  	%emph={square}, 
  	%emphstyle=\color{blue}\texttt,
  	%emph={[2]root,base},
  	%emphstyle={[2]\color{yac}\texttt},
}
% Farben definieren
\lstset{
	keywordstyle=\bfseries\color{purple},
	commentstyle=\itshape\color{dunkelgrau},
	stringstyle=\color{darkblue},
}

% URL Umbruch definieren
\expandafter\def\expandafter\UrlBreaks\expandafter{\UrlBreaks\do\a%
\do\b\do\c\do\d\do\e\do\f\do\g\do\h\do\i\do\j\do\k\do\l\do\m\do\n%
\do\o\do\p\do\q\do\r\do\s\do\t\do\u\do\v\do\w\do\x\do\y\do\z\do\&}

% Tikz-Einstellungen
%\PreviewEnvironment{tikzpicture}
%\setlength\PreviewBorder{5pt}

% Änderung der KOMA-Befehle für die Überschriften (andere Gestaltung):
%\definecolor{partcolor}{rgb}{.176,0,.367}
%\definecolor{chaptercolor}{rgb}{.176,0,.367}
\definecolor{sectioncolor}{rgb}{.176,0,.367}
%\definecolor{paragraphcolor}{rgb}{.176,0,.367}
%\addtokomafont{part}{\normalfont\Huge\bfseries\sffamily\color{partcolor}}
%\addtokomafont{partnumber}{\normalfont\Huge\bfseries\sffamily\color{partcolor}}
%\addtokomafont{chapter}{\normalfont\Huge\bfseries\sffamily\color{chaptercolor}\hspace*{-1cm}}
%\addtokomafont{section}{\color{sectioncolor}\vspace*{-0.4cm}}
%\addtokomafont{subsection}{\color{sectioncolor}\vspace*{-0.3cm}}
%\addtokomafont{subsubsection}{\color{sectioncolor}}
%\addtokomafont{paragraph}{\normalfont\small\bfseries\sffamily\color{paragraphcolor}}
%\addtokomafont{subparagraph}{\normalfont\small\itshape\sffamily\color{paragraphcolor}}

%%% marginrule
\colorlet{marginColor}{black!5}
\newcommand\marginrule[1]{%
  \begingroup
  \color{blue}\arrayrulewidth1pt
  \fboxsep0pt\fboxrule0pt
  \marginnote{%
    \fcolorbox{marginColor}{marginColor}{%
      \parbox[t]{\marginparwidth}{%
      \ifodd\value{page}%  
      \begin{tabular}{|@{~}p{\marginparwidth}@{}}
        \begin{spacing}{1}
          \RaggedRight\mbox{}\footnotesize\itshape\color{blue}#1
          \vskip-2.5\baselineskip
        \end{spacing}
      \end{tabular}
      \else
      \begin{tabular}{@{}p{\marginparwidth-3pt}@{~}|}
        \begin{spacing}{1}
          \RaggedLeft\mbox{}\footnotesize\itshape\color{blue}#1
          \vskip-2.5\baselineskip
        \end{spacing}
      \end{tabular}
      \fi
    }}%
  }%
  \endgroup
}

% Verweise
%\crefname { equation }{ Gleichung }{ Gleichungen }
\crefname { definition }{ Definition }{ Definitionen }
\crefname { pic }{ Abbildung }{ Abbildungen }
%\crefname { table }{ Tabelle }{ Tabelle }
%\crefname { figure }{ Figur }{ Figuren }
\crefformat { equation }{ Gleichung ~#2(#1) #3}
\crefformat { shadedefinitions }{#2Definition~#1#3}
\crefformat { pic }{ Abbildung ~#2(#1) #3}
%\crefformat { table }{ Tabelle ~#2(#1) #3}
%\crefformat { figure }{ Figur ~#2(#1) #3}
%\crefrangeformat { equation }{ Gleichungen ~#3(#1) #4 bis ~#5(#2) #6}
%\crefmultiformat { equation }{ Gleichungen ~#2(#1) #3} { und ~#2(#1) #3}{ , #2(#1) #3}{ und ~#2(#1) #3}

% Schusterjungen und Hurenkinder vermeiden
\clubpenalty = 10000
\widowpenalty = 10000
\displaywidowpenalty = 10000

% Mathematik-Layout
\newshadetheorem{thms}{Satz}
\newshadetheorem{beweise}{Beweis}
\newshadetheorem{definitions}{Definition}[section]
\newshadetheorem{cors}{Korollar}
\newshadetheorem{zuss}{Zusammenfassung}
\newshadetheorem{axioms}{Axiom}
\newshadetheorem{bsps}{Beispiel}

\newenvironment{thm}[1][]{%
  \definecolor{shadethmcolor}{rgb}{.9,.9,.95}%
  \definecolor{shaderulecolor}{rgb}{0.0,0.0,0.4}%
  \setlength{\shadeboxrule}{1.5pt}%
  \begin{thms}[#1]%\hspace*{1mm}%
}{\end{thms}{\hspace{\stretch{1}}\rule{1ex}{1ex}}
}

\newenvironment{beweis}[1][]{%
  \definecolor{shadethmcolor}{HTML}{EAEAEA}%
%  \definecolor{shaderulecolor}{rgb}{0.0,0.0,0.4}%
%  \setlength{\shadeboxrule}{1.5pt}%
  \begin{beweise}[#1]%\hspace*{1mm}%
}{\end{beweise}{\hfill}\rule{1ex}{1ex}}

\newenvironment{definition}[1][]{%
	\definecolor{shadethmcolor}{HTML}{EAEAEA}
  \definecolor{shaderulecolor}{HTML}{525252}
  \setlength{\shadeboxrule}{0.5pt}%
  \begin{definitions}[#1]\hspace*{1mm}%
}{\end{definitions}}

%\newenvironment{bsp}[1][]{%
%  \definecolor{shadethmcolor}{rgb}{.9,.9,.95}%
%  \definecolor{shaderulecolor}{rgb}{0.0,0.0,0.4}%
%  \setlength{\shadeboxrule}{1.5pt}%
%  \begin{bsps}[#1]\hspace*{1mm}%
%}{\end{bsps}}

%
%   Für Sätze, Definitionen und Beispiele
%   ein * unterdrückt eine Nummerierung
%
\newtheorem*{fakt}{Fakt}
\newtheorem{fazit}{Fazit}
\newtheorem{satz}{Satz}
\newtheorem{theo}{Theorem}
\newtheorem{lemma}[satz]{Lemma}
% Bemerke hier das zusätzliche [satz]: Dies bewirkt,
% dass ein Satz und ein Lemma den gleichen Zähler verwenden
\newtheorem{chara}{Charakterisierung}

% Definitionen haben einen anderen Stil als Sätze
\theoremstyle{definition}
\newtheorem{defi}{Definition}[section]
\newtheorem{termi}{Terminologie}

% Remark hat ebenfalls einen anderen Stil
\theoremstyle{remark}
\newtheorem*{beh}{Behauptung}
\newtheorem{bem}{Bemerkung}
\newtheorem*{rem}{Bemerkung}
\newtheorem*{beo}{Beobachtung}
\newtheorem*{folg}{Folgerung}
\newtheorem{korr}{Korollar}
\newtheorem*{bsp}{Beispiel}
\newtheorem*{bspe}{Beispiele}

% Dies ist die einfachste Methode, den Beweisnamen umzunennen
\renewcommand{\proofname}{Beweis}

% Wer Beweis gerne anders formatiert haben will, muss folgendes benutzen:
\newcommand{\proofend}{\begin{flushright}$\Box$\end{flushright}}
%\renewcommand{\proof}{\textbf{Beweis}: }

\newcommand{\proofcon}{\textbf{Beweis} (durch Kontraposition):}
\newcommand{\proofind}{\textbf{Beweis} (indirekt): }
\newcommand{\induction}[1]{\textbf{Beweis} (durch vollständige Induktion nach $#1$): }
\newcommand{\inda}[2]{\textit{Induktionsanfang ($#1=#2$): }}
\newcommand{\indv}{\textit{Induktionsvoraussetzung ($\star$): }}
\newcommand{\inds}[1]{\textit{Induktionsschritt ($#1\rightsquigarrow #1+1$): }}

%
% wenn man nicht immer "\begin{proof}[Beweis]" eingeben will: "\begin{bew}":
%
\newenvironment{bew}
	{\begin{proof}[\textbf{Beweis}: ]}
	{\end{proof}}
\newenvironment{direkt}
	{\begin{proof}[\textbf{Beweis} (direkt): ]}
	{\end{proof}}
%\newenvironment{induktion}[1]
%	{\begin{proof}[\induction[#1]]}
%	{\end{proof}}
\newenvironment{induktion}
	{\begin{proof}[\textbf{Beweis} (durch vollständige Induktion): ]}
	{\end{proof}}
\newenvironment{kontrapos}
	{\begin{proof}[\proofcon]}
	{\end{proof}}
\newenvironment{indirekt}
	{\begin{proof}[\proofind]}
	{\end{proof}}

\newenvironment{alphenumi}%
    {\renewcommand{\labelenumi}{\alph{enumi})}\begin{enumerate}}
    {\end{enumerate}}
\newenvironment{adenumi}%
    {\renewcommand{\labelenumi}{{ad} \arabic{enumi})}\begin{enumerate}}
    {\end{enumerate}}

